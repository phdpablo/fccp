% Options for packages loaded elsewhere
\PassOptionsToPackage{unicode}{hyperref}
\PassOptionsToPackage{hyphens}{url}
\PassOptionsToPackage{dvipsnames,svgnames,x11names}{xcolor}
%
\documentclass[
  a4paper,
]{book}

\usepackage{amsmath,amssymb}
\usepackage{iftex}
\ifPDFTeX
  \usepackage[T1]{fontenc}
  \usepackage[utf8]{inputenc}
  \usepackage{textcomp} % provide euro and other symbols
\else % if luatex or xetex
  \usepackage{unicode-math}
  \defaultfontfeatures{Scale=MatchLowercase}
  \defaultfontfeatures[\rmfamily]{Ligatures=TeX,Scale=1}
\fi
\usepackage{lmodern}
\ifPDFTeX\else  
    % xetex/luatex font selection
\fi
% Use upquote if available, for straight quotes in verbatim environments
\IfFileExists{upquote.sty}{\usepackage{upquote}}{}
\IfFileExists{microtype.sty}{% use microtype if available
  \usepackage[]{microtype}
  \UseMicrotypeSet[protrusion]{basicmath} % disable protrusion for tt fonts
}{}
\makeatletter
\@ifundefined{KOMAClassName}{% if non-KOMA class
  \IfFileExists{parskip.sty}{%
    \usepackage{parskip}
  }{% else
    \setlength{\parindent}{0pt}
    \setlength{\parskip}{6pt plus 2pt minus 1pt}}
}{% if KOMA class
  \KOMAoptions{parskip=half}}
\makeatother
\usepackage{xcolor}
\usepackage[paperwidth=8.27in,paperheight=11.69in,left=1.25in,textwidth=
5.25in,top=1.00in,textheight=8.25in]{geometry}
\setlength{\emergencystretch}{3em} % prevent overfull lines
\setcounter{secnumdepth}{5}
% Make \paragraph and \subparagraph free-standing
\ifx\paragraph\undefined\else
  \let\oldparagraph\paragraph
  \renewcommand{\paragraph}[1]{\oldparagraph{#1}\mbox{}}
\fi
\ifx\subparagraph\undefined\else
  \let\oldsubparagraph\subparagraph
  \renewcommand{\subparagraph}[1]{\oldsubparagraph{#1}\mbox{}}
\fi


\providecommand{\tightlist}{%
  \setlength{\itemsep}{0pt}\setlength{\parskip}{0pt}}\usepackage{longtable,booktabs,array}
\usepackage{calc} % for calculating minipage widths
% Correct order of tables after \paragraph or \subparagraph
\usepackage{etoolbox}
\makeatletter
\patchcmd\longtable{\par}{\if@noskipsec\mbox{}\fi\par}{}{}
\makeatother
% Allow footnotes in longtable head/foot
\IfFileExists{footnotehyper.sty}{\usepackage{footnotehyper}}{\usepackage{footnote}}
\makesavenoteenv{longtable}
\usepackage{graphicx}
\makeatletter
\def\maxwidth{\ifdim\Gin@nat@width>\linewidth\linewidth\else\Gin@nat@width\fi}
\def\maxheight{\ifdim\Gin@nat@height>\textheight\textheight\else\Gin@nat@height\fi}
\makeatother
% Scale images if necessary, so that they will not overflow the page
% margins by default, and it is still possible to overwrite the defaults
% using explicit options in \includegraphics[width, height, ...]{}
\setkeys{Gin}{width=\maxwidth,height=\maxheight,keepaspectratio}
% Set default figure placement to htbp
\makeatletter
\def\fps@figure{htbp}
\makeatother
% definitions for citeproc citations
\NewDocumentCommand\citeproctext{}{}
\NewDocumentCommand\citeproc{mm}{%
  \begingroup\def\citeproctext{#2}\cite{#1}\endgroup}
\makeatletter
 % allow citations to break across lines
 \let\@cite@ofmt\@firstofone
 % avoid brackets around text for \cite:
 \def\@biblabel#1{}
 \def\@cite#1#2{{#1\if@tempswa , #2\fi}}
\makeatother
\newlength{\cslhangindent}
\setlength{\cslhangindent}{1.5em}
\newlength{\csllabelwidth}
\setlength{\csllabelwidth}{3em}
\newenvironment{CSLReferences}[2] % #1 hanging-indent, #2 entry-spacing
 {\begin{list}{}{%
  \setlength{\itemindent}{0pt}
  \setlength{\leftmargin}{0pt}
  \setlength{\parsep}{0pt}
  % turn on hanging indent if param 1 is 1
  \ifodd #1
   \setlength{\leftmargin}{\cslhangindent}
   \setlength{\itemindent}{-1\cslhangindent}
  \fi
  % set entry spacing
  \setlength{\itemsep}{#2\baselineskip}}}
 {\end{list}}
\usepackage{calc}
\newcommand{\CSLBlock}[1]{\hfill\break\parbox[t]{\linewidth}{\strut\ignorespaces#1\strut}}
\newcommand{\CSLLeftMargin}[1]{\parbox[t]{\csllabelwidth}{\strut#1\strut}}
\newcommand{\CSLRightInline}[1]{\parbox[t]{\linewidth - \csllabelwidth}{\strut#1\strut}}
\newcommand{\CSLIndent}[1]{\hspace{\cslhangindent}#1}

\makeatletter
\@ifpackageloaded{bookmark}{}{\usepackage{bookmark}}
\makeatother
\makeatletter
\@ifpackageloaded{caption}{}{\usepackage{caption}}
\AtBeginDocument{%
\ifdefined\contentsname
  \renewcommand*\contentsname{Índice}
\else
  \newcommand\contentsname{Índice}
\fi
\ifdefined\listfigurename
  \renewcommand*\listfigurename{Lista de Figuras}
\else
  \newcommand\listfigurename{Lista de Figuras}
\fi
\ifdefined\listtablename
  \renewcommand*\listtablename{Lista de Tabelas}
\else
  \newcommand\listtablename{Lista de Tabelas}
\fi
\ifdefined\figurename
  \renewcommand*\figurename{Figura}
\else
  \newcommand\figurename{Figura}
\fi
\ifdefined\tablename
  \renewcommand*\tablename{Tabela}
\else
  \newcommand\tablename{Tabela}
\fi
}
\@ifpackageloaded{float}{}{\usepackage{float}}
\floatstyle{ruled}
\@ifundefined{c@chapter}{\newfloat{codelisting}{h}{lop}}{\newfloat{codelisting}{h}{lop}[chapter]}
\floatname{codelisting}{Listagem}
\newcommand*\listoflistings{\listof{codelisting}{Lista de Listagens}}
\makeatother
\makeatletter
\makeatother
\makeatletter
\@ifpackageloaded{caption}{}{\usepackage{caption}}
\@ifpackageloaded{subcaption}{}{\usepackage{subcaption}}
\makeatother
\ifLuaTeX
\usepackage[bidi=basic]{babel}
\else
\usepackage[bidi=default]{babel}
\fi
\babelprovide[main,import]{portuguese}
% get rid of language-specific shorthands (see #6817):
\let\LanguageShortHands\languageshorthands
\def\languageshorthands#1{}
\ifLuaTeX
  \usepackage{selnolig}  % disable illegal ligatures
\fi
\usepackage{bookmark}

\IfFileExists{xurl.sty}{\usepackage{xurl}}{} % add URL line breaks if available
\urlstyle{same} % disable monospaced font for URLs
\hypersetup{
  pdftitle={Finanças Corporativas de Curto Prazo},
  pdfauthor={Pablo Rogers},
  pdflang={pt},
  colorlinks=true,
  linkcolor={Maroon},
  filecolor={Maroon},
  citecolor={Blue},
  urlcolor={Blue},
  pdfcreator={LaTeX via pandoc}}

\title{Finanças Corporativas de Curto Prazo}
\author{Pablo Rogers}
\date{16 de junho de 2025}

\begin{document}
\frontmatter
\maketitle

\renewcommand*\contentsname{Índice}
{
\hypersetup{linkcolor=}
\setcounter{tocdepth}{2}
\tableofcontents
}
\mainmatter
\bookmarksetup{startatroot}

\chapter*{🏢 O Curso}\label{sec-home}
\addcontentsline{toc}{chapter}{🏢 O Curso}

\markboth{🏢 O Curso}{🏢 O Curso}

Página da disciplina \textbf{``Finanças Corporativas I''} do curso da
\href{https://www.facic.ufu.br/}{Faculdade de Ciência Contábeis} (FACIC)
da \href{https://ufu.br/}{Universidade Federal de Uberlândia} (UFU).
Aqui você encontrará informações sobre o programa do curso, materiais
para seu acompanhamento e sugestões de leituras sobre \textbf{Finanças
Corporativas de Curto Prazo} (artigos, notas de aulas, blogs, vídeos,
etc.).

\section*{Sobre o professor}\label{sec-instrutor}

\markright{Sobre o professor}

A disciplina é ministrada e mantida nesse hub por mim, Pablo Rogers 😉,
doutor em Administração pela Universidade de São Paulo (FEA/USP) e
professor de finanças e métodos quantitativos desde 2005 na UFU. Em meu
\href{https://phdpablo.com/}{site pessoal} você encontrará mais detalhes
sobre minhas formações, competências, trajetória e projetos.

\section*{Objetivos}\label{sec-about}

\markright{Objetivos}

O curso tem como objetivo apresentar os principais conceitos e práticas
de finanças corporativas de curto prazo. A disciplina visa prover aos
alunos uma visão teórica e prática da \textbf{Administração do Capital
de Giro} como base fundamental para o planejamento e controle financeiro
do curto prazo. Especificamente, ao final do curso pretende-se que o
aluno:

\begin{itemize}
\item
  Compreenda as teorias que embasam a gestão do capital de curto prazo;
\item
  Entenda a dinâmica da gestão do capital de giro;
\item
  Conheça as estratégias e modelos da gestão do caixa;
\item
  Compreenda a gestão de valores a receber, suas políticas e riscos
  envolvidos;
\item
  Assimile os aspectos gerais da gestão de estoques e seus modelos de
  análise.
\end{itemize}

\section*{Programa}\label{sec-programa}

\markright{Programa}

A ementa oficial da disciplina encontra-se
\href{https://www.facic.ufu.br/system/files/conteudo/28fagen39532_financas_corporativas_i.pdf}{aqui}.
O Plano de Ensino aprovado pela coordenação da FACIC/UFU pode ser
acessado no \href{https://moodle.ufu.br/login/index.php}{Moodle}, onde
materemos a comunicação e organização das avaliações. Em linhas gerais o
programa do curso versará sobre os seguintes conteúdos:

\begin{enumerate}
\def\labelenumi{\arabic{enumi}.}
\item
  Introdução às Finanças Corporativas de Curto Prazo (FCCP)

  Relação Risco e Retorno em Finanças

  Gestão de Curto Prazo x Gestão de Longo Prazo

  Teorias de Finanças
\item
  Administração Financeira do Curto Prazo (Capital de Giro)

  Conceitos de Capital de Giro

  Dinâmica Empresarial: Análise dos Ciclos Operacional e Financeiro

  Investimento em Capital de Giro

  Financiamento do Capital de Giro

  Necessidade de Investimento em Giro (NIG)
\item
  Administração de Caixa

  Razões da demanda de moeda e manutenção de caixa

  Ciclo de caixa e controle de seu saldo

  Modelos de administração de caixa
\item
  Administração de Valores a Receber (Recebíveis)

  Avaliação do risco de crédito

  Elementos de uma política geral de crédito
\item
  Administração de Estoques

  Aspectos básicos dos estoques

  Modelos de análise e controle dos estoques
\end{enumerate}

\section*{Metodologia}\label{sec-method}

\markright{Metodologia}

O material do curso organizado nesse repositório refere-se ao roteiro
estruturado (enredo) de parte que discutiremos nas aulas presenciais e
conteúdos adicionais (bibliografia, notas de aulas, links, dicas de
vídeos, etc). Na sala de aula teremos discussões conceituais e
resoluções de exercícios, e por aqui, num primeiro momento, focarei em
introduzir os \textbf{conceitos basilares da FCCP}.

A proposta do curso busca seguir de perto a mensagem de Dogucu \&
Çetinkaya-Rundel (2022). Nesse artigo as autoras abordam a importância
da reprodutibilidade na pesquisa e ensino. Elas recomendam que os
professores-pesquisadores adotem fluxos de trabalho reprodutíveis em
suas pesquisas e ensinem esses fluxos de trabalho aos seus alunos. Elas
propõem uma dimensão para as práticas de reprodutibilidade, focada
exclusivamente nas ferramentas para o ensino (todos os materiais de
ensino devem ser computacionalmente reprodutíveis, bem documentados e
abertos).

\section*{Bibliografia}\label{sec-biblio}

\markright{Bibliografia}

A literatura de finanças é vasta. No Brasil, temos vários bons manuais
em língua portuguesa. Muitos livros-textos são traduções de autores
americanos, ou seja, conteúdo ambientado em um mercado diferente do
nosso. No entanto, existem alguns manuais de autores brasileiros, cujo
conteúdo é adaptado para o contexto nacional. Vamos fazer uso dos dois!
😉

Geralmente, esses manuais percorrem diversos assuntos de finanças,
entretanto, nosso foco será na \textbf{FCCP}. Os outros assuntos serão
tratados em disciplinas correlatas: Matemática Financeira, Finanças
Corporativas II (Longo Prazo), Governança Corporativa, Avaliação
Econômica de Empresas e Mercado de Capitais. Sem falar das áreas
correlatas, tais como Economia (Micro e Macroeconomia), Matemática,
Estatística e Ciência da Computação. Na verdade, o conteúdo do curso de
Ciências Contábeis, no meu entender, é aquele que talvez dá a melhor
base para a formação de \textbf{Administrador Financeiro}, até mesmo
mais que o próprio curso de Administração 🤐🤫.

Como bibliografia base para os fundamentos do curso, utilizaremos as
recomendações da ementa oficial, e adotaremos as referências atualizadas
das bibliografias básica e complementar: Assaf Neto (2014), Gitman
(2010), Matias (2007), Ross et al. (2015) e Brealey et al. (2013).

\section*{Licença}\label{licenuxe7a}

\markright{Licença}

Finanças Corporativas de Curto Prazo by Pablo Rogers is licensed under
CC BY-NC-SA 4.0

\bookmarksetup{startatroot}

\chapter*{📇 Pré-requisitos}\label{sec-prework}
\addcontentsline{toc}{chapter}{📇 Pré-requisitos}

\markboth{📇 Pré-requisitos}{📇 Pré-requisitos}

O único pré-requisito para a disciplina é \textbf{boa vontade e fazer as
pré-leituras indicadas antes de cada aula}. Para isso, sempre monitorar
nosso \href{http://fccp.phdpablo.com/00-schedule.html}{Cronograma}, para
não perder nenhum prazo, é importante 🤢. Diferente da disciplina de
Matemática Financeira (e/ou Análise de Investimentos) que, geralmente,
vem antes do conteúdo de finanças de curto prazo nos currículos
programáticos das universidades brasileiras, essa disciplina depende
``mais'' de leituras das teorias (conceitos) subjacentes.

De qualquer forma, creio ser importante situar os alunos sobre os
assuntos (disciplinas) correlatos que perfaz (vem antes e depois) o
conteúdo típico de finanças em currículos da área de negócios.
Especialmente, aqui na FACIC/UFU temos o seguinte panorama:

\begin{longtable}[]{@{}
  >{\raggedright\arraybackslash}p{(\columnwidth - 6\tabcolsep) * \real{0.2466}}
  >{\raggedright\arraybackslash}p{(\columnwidth - 6\tabcolsep) * \real{0.2466}}
  >{\raggedright\arraybackslash}p{(\columnwidth - 6\tabcolsep) * \real{0.2466}}
  >{\raggedright\arraybackslash}p{(\columnwidth - 6\tabcolsep) * \real{0.2603}}@{}}
\toprule\noalign{}
\begin{minipage}[b]{\linewidth}\raggedright
Disciplina
\end{minipage} & \begin{minipage}[b]{\linewidth}\raggedright
Foco Principal da Disciplina
\end{minipage} & \begin{minipage}[b]{\linewidth}\raggedright
Contribuição Chave para o Entendimento Financeiro Geral
\end{minipage} & \begin{minipage}[b]{\linewidth}\raggedright
Relação Específica e Relevância para Finanças Corporativas de Curto
Prazo (FCCP)
\end{minipage} \\
\midrule\noalign{}
\endhead
\bottomrule\noalign{}
\endlastfoot
\textbf{Matemática Financeira} & Juros, descontos, taxas, séries de
pagamento, valor do dinheiro no tempo, análise básica de investimentos.
& Fornece as ferramentas quantitativas fundamentais para toda a análise
financeira. & Essencial para calcular custos de financiamento de curto
prazo, avaliar descontos, analisar fluxos de caixa do capital de giro.
Base para todas as decisões quantitativas em FCCP. \\
\textbf{Finanças Corporativas (I) de Curto Prazo} & Gestão do capital de
giro (caixa, contas a receber, estoques), ciclo de conversão de caixa,
financiamento de curto prazo. & Desenvolve a compreensão da gestão
financeira operacional e da manutenção da liquidez da empresa. & É o
foco principal desta disciplina; estabelece os fundamentos da gestão
financeira diária. \\
\textbf{Finanças Corporativas (II) de Longo Prazo} & Estrutura de
capital, custo de capital, política de dividendos, orçamento de capital
avançado, fusões e aquisições. & Aprofunda nas decisões financeiras
estratégicas de longo prazo que moldam o futuro e o valor da empresa. &
A saúde financeira de curto prazo sustenta a capacidade de planejar e
executar estratégias de longo prazo. \\
\textbf{Avaliação Econômica de Empresas} & Métodos de avaliação (FCD,
múltiplos), goodwill, ativos intangíveis, gestão baseada em valor. &
Capacita na determinação do valor de uma empresa, crucial para decisões
de investimento, fusões e aquisições. & A gestão eficiente do capital de
giro (FCCP) impacta diretamente os fluxos de caixa, que são inputs chave
nos modelos de avaliação. A GBV conecta decisões de curto prazo ao valor
da empresa. \\
\textbf{Governança Corporativa} & Princípios (transparência, equidade,
accountability), mecanismos de controle, ética, relação com
stakeholders. & Estabelece o arcabouço para a tomada de decisão ética e
responsável, visando proteger os interesses dos investidores. & As
decisões de FCCP devem ser tomadas dentro de um sistema de boa
governança. Controles internos sobre o capital de giro são cruciais para
prevenir fraudes e má gestão. \\
\textbf{Mercado de Capitais} & Sistema Financeiro Nacional, mercado de
crédito, títulos (renda fixa/variável), mercado de ações, derivativos. &
Proporciona o entendimento do ambiente financeiro externo onde as
empresas captam recursos e investidores aplicam seu capital. & O mercado
de capitais influencia o custo das fontes de financiamento de curto
prazo e oferece opções para aplicação de caixa excedente. Algumas
empresas podem acessar o mercado para financiamento de curto prazo. \\
\end{longtable}

Mas claro, no limite, podemos considerar que todo conteúdo da disciplina
FCCP está conectado com outras áreas do conhecimento, tal como
transcorre nossa formação num curso típico da área de negócios
(economia, administração, contabilidade, etc.):

\begin{longtable}[]{@{}
  >{\raggedright\arraybackslash}p{(\columnwidth - 4\tabcolsep) * \real{0.2500}}
  >{\raggedright\arraybackslash}p{(\columnwidth - 4\tabcolsep) * \real{0.4028}}
  >{\raggedright\arraybackslash}p{(\columnwidth - 4\tabcolsep) * \real{0.3472}}@{}}
\toprule\noalign{}
\begin{minipage}[b]{\linewidth}\raggedright
Disciplina Relacionada
\end{minipage} & \begin{minipage}[b]{\linewidth}\raggedright
Principais Contribuições para Finanças Corporativas de Curto Prazo
\end{minipage} & \begin{minipage}[b]{\linewidth}\raggedright
Exemplos de Aplicação Prática em Finanças de Curto Prazo
\end{minipage} \\
\midrule\noalign{}
\endhead
\bottomrule\noalign{}
\endlastfoot
\textbf{Contabilidade} & Fornecimento de dados financeiros (DRE,
Balanço, DFC), princípios de mensuração e reconhecimento, base para
análise de custos e desempenho. & Análise de índices de liquidez a
partir do Balanço Patrimonial; uso da DRE para calcular a margem de
contribuição; elaboração do Fluxo de Caixa para gestão da tesouraria. \\
\textbf{Economia (Micro e Macro)} & Teoria dos preços, análise de
custos, estruturas de mercado (Micro); taxas de juros, inflação, ciclos
econômicos, políticas fiscais e monetárias (Macro). & Definição de
política de preços considerando a elasticidade da demanda; impacto da
taxa Selic no custo de empréstimos de curto prazo; ajuste de estoques
com base na previsão de crescimento do PIB. \\
\textbf{Matemática} & Lógica, álgebra, cálculo (implícito em modelos de
otimização), e especificamente Matemática Financeira (valor do dinheiro
no tempo, taxas, etc.). & Cálculo do custo efetivo de um desconto
financeiro oferecido por fornecedor; determinação do ponto de
equilíbrio; modelagem do saldo ótimo de caixa. \\
\textbf{Estatística} & Teoria da probabilidade, análise de regressão,
previsão de séries temporais, amostragem, testes de hipóteses,
estatística descritiva. & Previsão de vendas para planejar estoques;
análise de risco de crédito de clientes; cálculo da média e desvio
padrão de prazos de pagamento; uso da Curva ABC para classificar
clientes ou produtos. \\
\end{longtable}

O que deve ficar claro é que, apesar de não haver pré-requisitos
formais, o conhecimento prévio em matemática financeira e contabilidade
é extremamente útil para o sucesso na disciplina. Além disso, a
compreensão dos conceitos econômicos e estatísticos pode enriquecer a
análise financeira e a tomada de decisões. Portanto, é recomendável que
os alunos tenham uma base sólida nessas áreas, caso deseja desempenhar
funções que se relacione com finanças corporativas de curto prazo. Fica
a dica!

A grande área de finanças abarca um vasto campo de estudo e prática que
se estende desde as decisões de um indivíduo até a complexa arquitetura
do sistema financeiro global. Tal formulação sugere que finanças atuam
como uma meta-disciplina, orquestrando e integrando conhecimentos de
diversas outras áreas -- como contabilidade, economia e estatística --
para otimizar os resultados financeiros e a sustentabilidade das
organizações. Assim, a disciplina de FCCP se posiciona como um elo
crucial nesse ecossistema, focando na gestão eficiente dos recursos
financeiros no dia a dia das empresas, garantindo não apenas a
sobrevivência, mas também o crescimento e a competitividade no mercado.

\bookmarksetup{startatroot}

\chapter*{📅 Cronograma}\label{sec-schedule}

\markboth{📅 Cronograma}{📅 Cronograma}

Planejamento dos dias (📅) e horários dos \textbf{conteúdos} (⏲️)
conforme nosso \emph{Plano de Ensino}. De uma forma geral, as aulas
presenciais da disciplina ocorrerão às sexta-feiras das 19:00 às 22:30
durante o período 09/06/2025 a 29/09/2025. O cronograma detalhado, com
as datas de todas as nossas avaliações, encontra-se em nosso
\textbf{Plano de Ensino no Moodle}.

A ideia aqui nessa página é apenas roterizar o conteúdo da disciplina
FCCP, com direcionamento de materiais suplementares, para o aluno
\textbf{organizar suas leituras prévias}.

Na seção de cada um dos módulos (tópicos) do conteúdo programático temos
a indicação da \textbf{bibliografia básica}. Quando disponível, por
aqui, poderás acessar os slides utilizados nas aulas (🗣️), aulas
gravadas ou indicações de vídeo (🎥{]} e \textbf{leituras
complementares} sobre os conteúdos (📓).

\begin{longtable}[]{@{}lcc@{}}
\toprule\noalign{}
Aula/Conteúdo & Data & Material Complementar \\
\midrule\noalign{}
\endhead
\bottomrule\noalign{}
\endlastfoot
Capítulo~\ref{sec-intro} & 📅20/06/25⏲️19:00 & 🗣🎥📓 \\
Capítulo~\ref{sec-giro} & 📅27/06/25⏲️19:00 & 🗣🎥📓 \\
Capítulo~\ref{sec-caixa} & 📅18/07/25⏲️19:00 & 🗣🎥📓 \\
Capítulo~\ref{sec-credito} & 📅01/08/25⏲️19:00 & 🗣🎥📓 \\
Capítulo~\ref{sec-estoque} & 📅19/08/25⏲️19:00 & 🗣🎥📓 \\
Capítulo~\ref{sec-aval} & 📅29/08/25⏲️19:00 & 🗣🎥📓 \\
\end{longtable}

\bookmarksetup{startatroot}

\chapter{Introdução à Finanças Corporativas}\label{sec-intro}

\bookmarksetup{startatroot}

\chapter{Capital de Giro}\label{sec-giro}

\bookmarksetup{startatroot}

\chapter{Administração do Caixa}\label{sec-caixa}

\bookmarksetup{startatroot}

\chapter{Administração de Recebíveis}\label{sec-credito}

\bookmarksetup{startatroot}

\chapter{Administração de Estoques}\label{sec-estoque}

\bookmarksetup{startatroot}

\chapter{Projetos em Grupo}\label{sec-aval}

\bookmarksetup{startatroot}

\chapter*{Referências}\label{referuxeancias}
\addcontentsline{toc}{chapter}{Referências}

\markboth{Referências}{Referências}

\phantomsection\label{refs}
\begin{CSLReferences}{1}{0}
\bibitem[\citeproctext]{ref-assafneto2014}
Assaf Neto, A. (2014). \emph{Finan{ç}as Corporativas e Valor} (7. ed.).
Atlas.

\bibitem[\citeproctext]{ref-brealey2013}
Brealey, R. A., Myers, S. C., \& Allen, F. (2013). \emph{Princ{í}pios de
Finan{ç}as Corporativas} (10. ed.). AMGH.

\bibitem[\citeproctext]{ref-dogucu2022}
Dogucu, M., \& Çetinkaya-Rundel, M. (2022). Tools and Recommendations
for Reproducible Teaching. \emph{Journal of Statistics and Data Science
Education}, \emph{30}(3), 251--260.
\url{https://doi.org/10.1080/26939169.2022.2138645}

\bibitem[\citeproctext]{ref-gitman2010}
Gitman, L. J. (2010). \emph{Princ{í}pios de Administra{ç}{ã}o
Financeira} (12. ed.). Pearson Prentice Hall.

\bibitem[\citeproctext]{ref-matias2007}
Matias, A. M. (Ed.). (2007). \emph{Finan{ç}as Corporativas de Curto
Prazo, Volume 1: A Gest{ã}o Do Valor Do Capital de Giro}. Atlas.

\bibitem[\citeproctext]{ref-ross2015}
Ross, S. A., Westerfield, R. W., Jaffe, J., \& Lamb, R. (2015).
\emph{Administra{ç}{ã}o Financeira} (10. ed.). AMGH.

\end{CSLReferences}


\backmatter

\end{document}
