% Options for packages loaded elsewhere
\PassOptionsToPackage{unicode}{hyperref}
\PassOptionsToPackage{hyphens}{url}
\PassOptionsToPackage{dvipsnames,svgnames,x11names}{xcolor}
%
\documentclass[
  a4paper,
]{book}

\usepackage{amsmath,amssymb}
\usepackage{iftex}
\ifPDFTeX
  \usepackage[T1]{fontenc}
  \usepackage[utf8]{inputenc}
  \usepackage{textcomp} % provide euro and other symbols
\else % if luatex or xetex
  \usepackage{unicode-math}
  \defaultfontfeatures{Scale=MatchLowercase}
  \defaultfontfeatures[\rmfamily]{Ligatures=TeX,Scale=1}
\fi
\usepackage{lmodern}
\ifPDFTeX\else  
    % xetex/luatex font selection
\fi
% Use upquote if available, for straight quotes in verbatim environments
\IfFileExists{upquote.sty}{\usepackage{upquote}}{}
\IfFileExists{microtype.sty}{% use microtype if available
  \usepackage[]{microtype}
  \UseMicrotypeSet[protrusion]{basicmath} % disable protrusion for tt fonts
}{}
\makeatletter
\@ifundefined{KOMAClassName}{% if non-KOMA class
  \IfFileExists{parskip.sty}{%
    \usepackage{parskip}
  }{% else
    \setlength{\parindent}{0pt}
    \setlength{\parskip}{6pt plus 2pt minus 1pt}}
}{% if KOMA class
  \KOMAoptions{parskip=half}}
\makeatother
\usepackage{xcolor}
\usepackage[paperwidth=8.27in,paperheight=11.69in,left=1.25in,textwidth=
5.25in,top=1.00in,textheight=8.25in]{geometry}
\setlength{\emergencystretch}{3em} % prevent overfull lines
\setcounter{secnumdepth}{5}
% Make \paragraph and \subparagraph free-standing
\ifx\paragraph\undefined\else
  \let\oldparagraph\paragraph
  \renewcommand{\paragraph}[1]{\oldparagraph{#1}\mbox{}}
\fi
\ifx\subparagraph\undefined\else
  \let\oldsubparagraph\subparagraph
  \renewcommand{\subparagraph}[1]{\oldsubparagraph{#1}\mbox{}}
\fi


\providecommand{\tightlist}{%
  \setlength{\itemsep}{0pt}\setlength{\parskip}{0pt}}\usepackage{longtable,booktabs,array}
\usepackage{calc} % for calculating minipage widths
% Correct order of tables after \paragraph or \subparagraph
\usepackage{etoolbox}
\makeatletter
\patchcmd\longtable{\par}{\if@noskipsec\mbox{}\fi\par}{}{}
\makeatother
% Allow footnotes in longtable head/foot
\IfFileExists{footnotehyper.sty}{\usepackage{footnotehyper}}{\usepackage{footnote}}
\makesavenoteenv{longtable}
\usepackage{graphicx}
\makeatletter
\def\maxwidth{\ifdim\Gin@nat@width>\linewidth\linewidth\else\Gin@nat@width\fi}
\def\maxheight{\ifdim\Gin@nat@height>\textheight\textheight\else\Gin@nat@height\fi}
\makeatother
% Scale images if necessary, so that they will not overflow the page
% margins by default, and it is still possible to overwrite the defaults
% using explicit options in \includegraphics[width, height, ...]{}
\setkeys{Gin}{width=\maxwidth,height=\maxheight,keepaspectratio}
% Set default figure placement to htbp
\makeatletter
\def\fps@figure{htbp}
\makeatother
% definitions for citeproc citations
\NewDocumentCommand\citeproctext{}{}
\NewDocumentCommand\citeproc{mm}{%
  \begingroup\def\citeproctext{#2}\cite{#1}\endgroup}
\makeatletter
 % allow citations to break across lines
 \let\@cite@ofmt\@firstofone
 % avoid brackets around text for \cite:
 \def\@biblabel#1{}
 \def\@cite#1#2{{#1\if@tempswa , #2\fi}}
\makeatother
\newlength{\cslhangindent}
\setlength{\cslhangindent}{1.5em}
\newlength{\csllabelwidth}
\setlength{\csllabelwidth}{3em}
\newenvironment{CSLReferences}[2] % #1 hanging-indent, #2 entry-spacing
 {\begin{list}{}{%
  \setlength{\itemindent}{0pt}
  \setlength{\leftmargin}{0pt}
  \setlength{\parsep}{0pt}
  % turn on hanging indent if param 1 is 1
  \ifodd #1
   \setlength{\leftmargin}{\cslhangindent}
   \setlength{\itemindent}{-1\cslhangindent}
  \fi
  % set entry spacing
  \setlength{\itemsep}{#2\baselineskip}}}
 {\end{list}}
\usepackage{calc}
\newcommand{\CSLBlock}[1]{\hfill\break\parbox[t]{\linewidth}{\strut\ignorespaces#1\strut}}
\newcommand{\CSLLeftMargin}[1]{\parbox[t]{\csllabelwidth}{\strut#1\strut}}
\newcommand{\CSLRightInline}[1]{\parbox[t]{\linewidth - \csllabelwidth}{\strut#1\strut}}
\newcommand{\CSLIndent}[1]{\hspace{\cslhangindent}#1}

\makeatletter
\@ifpackageloaded{tcolorbox}{}{\usepackage[skins,breakable]{tcolorbox}}
\@ifpackageloaded{fontawesome5}{}{\usepackage{fontawesome5}}
\definecolor{quarto-callout-color}{HTML}{909090}
\definecolor{quarto-callout-note-color}{HTML}{0758E5}
\definecolor{quarto-callout-important-color}{HTML}{CC1914}
\definecolor{quarto-callout-warning-color}{HTML}{EB9113}
\definecolor{quarto-callout-tip-color}{HTML}{00A047}
\definecolor{quarto-callout-caution-color}{HTML}{FC5300}
\definecolor{quarto-callout-color-frame}{HTML}{acacac}
\definecolor{quarto-callout-note-color-frame}{HTML}{4582ec}
\definecolor{quarto-callout-important-color-frame}{HTML}{d9534f}
\definecolor{quarto-callout-warning-color-frame}{HTML}{f0ad4e}
\definecolor{quarto-callout-tip-color-frame}{HTML}{02b875}
\definecolor{quarto-callout-caution-color-frame}{HTML}{fd7e14}
\makeatother
\makeatletter
\@ifpackageloaded{bookmark}{}{\usepackage{bookmark}}
\makeatother
\makeatletter
\@ifpackageloaded{caption}{}{\usepackage{caption}}
\AtBeginDocument{%
\ifdefined\contentsname
  \renewcommand*\contentsname{Índice}
\else
  \newcommand\contentsname{Índice}
\fi
\ifdefined\listfigurename
  \renewcommand*\listfigurename{Lista de Figuras}
\else
  \newcommand\listfigurename{Lista de Figuras}
\fi
\ifdefined\listtablename
  \renewcommand*\listtablename{Lista de Tabelas}
\else
  \newcommand\listtablename{Lista de Tabelas}
\fi
\ifdefined\figurename
  \renewcommand*\figurename{Figura}
\else
  \newcommand\figurename{Figura}
\fi
\ifdefined\tablename
  \renewcommand*\tablename{Tabela}
\else
  \newcommand\tablename{Tabela}
\fi
}
\@ifpackageloaded{float}{}{\usepackage{float}}
\floatstyle{ruled}
\@ifundefined{c@chapter}{\newfloat{codelisting}{h}{lop}}{\newfloat{codelisting}{h}{lop}[chapter]}
\floatname{codelisting}{Listagem}
\newcommand*\listoflistings{\listof{codelisting}{Lista de Listagens}}
\makeatother
\makeatletter
\makeatother
\makeatletter
\@ifpackageloaded{caption}{}{\usepackage{caption}}
\@ifpackageloaded{subcaption}{}{\usepackage{subcaption}}
\makeatother
\newcounter{quartocallouttipno}
\newcommand{\quartocallouttip}[1]{\refstepcounter{quartocallouttipno}\label{#1}}
\ifLuaTeX
\usepackage[bidi=basic]{babel}
\else
\usepackage[bidi=default]{babel}
\fi
\babelprovide[main,import]{portuguese}
% get rid of language-specific shorthands (see #6817):
\let\LanguageShortHands\languageshorthands
\def\languageshorthands#1{}
\ifLuaTeX
  \usepackage{selnolig}  % disable illegal ligatures
\fi
\usepackage{bookmark}

\IfFileExists{xurl.sty}{\usepackage{xurl}}{} % add URL line breaks if available
\urlstyle{same} % disable monospaced font for URLs
\hypersetup{
  pdftitle={Finanças Corporativas de Curto Prazo},
  pdfauthor={Pablo Rogers},
  pdflang={pt},
  colorlinks=true,
  linkcolor={Maroon},
  filecolor={Maroon},
  citecolor={Blue},
  urlcolor={Blue},
  pdfcreator={LaTeX via pandoc}}

\title{Finanças Corporativas de Curto Prazo}
\author{Pablo Rogers}
\date{13 de junho de 2025}

\begin{document}
\frontmatter
\maketitle

\renewcommand*\contentsname{Índice}
{
\hypersetup{linkcolor=}
\setcounter{tocdepth}{2}
\tableofcontents
}
\mainmatter
\bookmarksetup{startatroot}

\chapter*{O Curso 🏢}\label{sec-home}
\addcontentsline{toc}{chapter}{O Curso 🏢}

\markboth{O Curso 🏢}{O Curso 🏢}

Página da disciplina \textbf{``Finanças Corporativas I''} do curso da
\href{https://www.facic.ufu.br/}{Faculdade de Ciência Contábeis} (FACIC)
da \href{https://ufu.br/}{Universidade Federal de Uberlândia} (UFU).
Aqui você encontrará informações sobre o programa do curso, materiais
para seu acompanhamento e sugestões de leituras sobre \textbf{Finanças
Corporativas de Curto Prazo} (artigos, notas de aulas, blogs, vídeos,
etc.).

\section*{Sobre o professor}\label{sec-instrutor}

\markright{Sobre o professor}

A disciplina é ministrada e mantida nesse hub por mim, Pablo Rogers 😉,
doutor em Administração pela Universidade de São Paulo (FEA/USP) e
professor de finanças e métodos quantitativos desde 2005 na UFU. Em meu
\href{https://phdpablo.com/}{site pessoal} você encontrará mais detalhes
sobre minhas formações, competências, trajetória e projetos.

\section*{Objetivos}\label{sec-about}

\markright{Objetivos}

O curso tem como objetivo apresentar os principais conceitos e práticas
de finanças corporativas de curto prazo. A disciplina visa prover aos
alunos uma visão teórica e prática da \textbf{Administração do Capital
de Giro} como base fundamental para o planejamento e controle financeiro
do curto prazo. Especificamente, ao final do curso pretende-se que o
aluno:

\begin{itemize}
\item
  Compreenda as teorias que embasam a gestão do capital de curto prazo;
\item
  Entenda a dinâmica da gestão do capital de giro;
\item
  Conheça as estratégias e modelos da gestão do caixa;
\item
  Compreenda a gestão de valores a receber, suas políticas e riscos
  envolvidos;
\item
  Assimile os aspectos gerais da gestão de estoques e seus modelos de
  análise.
\end{itemize}

\section*{Programa}\label{sec-programa}

\markright{Programa}

A ementa oficial da disciplina encontra-se
\href{https://www.facic.ufu.br/system/files/conteudo/28fagen39532_financas_corporativas_i.pdf}{aqui}.
O Plano de Ensino aprovado pela coordenação da FACIC/UFU pode ser
acessado no \href{https://moodle.ufu.br/login/index.php}{Moodle}, onde
materemos a comunicação e organização das avaliações. Em linhas gerais o
programa do curso versará sobre os seguintes conteúdos:

\begin{enumerate}
\def\labelenumi{\arabic{enumi}.}
\item
  Introdução às Finanças Corporativas de Curto Prazo (FCCP)

  Relação Risco e Retorno em Finanças

  Gestão de Curto Prazo x Gestão de Longo Prazo

  Teorias de Finanças
\item
  Administração Financeira do Curto Prazo (Capital de Giro)

  Conceitos de Capital de Giro

  Dinâmica Empresarial: Análise dos Ciclos Operacional e Financeiro

  Investimento em Capital de Giro

  Financiamento do Capital de Giro

  Necessidade de Investimento em Giro (NIG)
\item
  Administração de Caixa

  Razões da demanda de moeda e manutenção de caixa

  Ciclo de caixa e controle de seu saldo

  Modelos de administração de caixa
\item
  Administração de Valores a Receber (Recebíveis)

  Avaliação do risco de crédito

  Elementos de uma política geral de crédito
\item
  Administração de Estoques

  Aspectos básicos dos estoques

  Modelos de análise e controle dos estoques
\end{enumerate}

\section*{Metodologia}\label{sec-method}

\markright{Metodologia}

O material do curso organizado nesse repositório refere-se ao roteiro
estruturado (enredo) de parte que discutiremos nas aulas presenciais e
conteúdos adicionais (bibliografia, notas de aulas, links, dicas de
vídeos, etc). Na sala de aula teremos discussões conceituais e
resoluções de exercícios, e por aqui, num primeiro momento, focarei em
introduzir os \textbf{conceitos basilares da FCCP}.

A proposta do curso busca seguir de perto a mensagem de Dogucu \&
Çetinkaya-Rundel (2022). Nesse artigo as autoras abordam a importância
da reprodutibilidade na pesquisa e ensino. Elas recomendam que os
professores-pesquisadores adotem fluxos de trabalho reprodutíveis em
suas pesquisas e ensinem esses fluxos de trabalho aos seus alunos. Elas
propõem uma dimensão para as práticas de reprodutibilidade, focada
exclusivamente nas ferramentas para o ensino (todos os materiais de
ensino devem ser computacionalmente reprodutíveis, bem documentados e
abertos).

\section*{Bibliografia}\label{sec-biblio}

\markright{Bibliografia}

A literatura de finanças é vasta. No Brasil, temos vários bons manuais
em língua portuguesa. Muitos livros-textos são traduções de autores
americanos, ou seja, conteúdo ambientado em um mercado diferente do
nosso. No entanto, existem alguns manuais de autores brasileiros, cujo
conteúdo é adaptado para o contexto nacional. Vamos fazer uso dos dois!
😉

Geralmente, esses manuais percorrem diversos assuntos de finanças,
entretanto, nosso foco será na \textbf{FCCP}. Os outros assuntos serão
tratados em disciplinas correlatas: Matemática Financeira, Finanças
Corporativas II (Longo Prazo), Governança Corporativa, Avaliação
Econômica de Empresas e Mercado de Capitais. Sem falar das áreas
correlatas, tais como Economia (Micro e Macroeconomia), Matemática,
Estatística e Ciência da Computação. Na verdade, o conteúdo do curso de
Ciências Contábeis, no meu entender, é aquele que talvez dá a melhor
base para a formação de \textbf{Administrador Financeiro}, até mesmo
mais que o próprio curso de Administração 🤐🤫.

Como bibliografia base para os fundamentos do curso, utilizaremos as
recomendações da ementa oficial, e adotaremos as referências atualizadas
das bibliografias básica e complementar: Assaf Neto (2014), Gitman
(2010), Matias (2007), Ross et al. (2015) e Brealey et al. (2013).

\section*{Licença}\label{licenuxe7a}

\markright{Licença}

Finanças Corporativas de Curto Prazo by Pablo Rogers is licensed under
CC BY-NC-SA 4.0

\bookmarksetup{startatroot}

\chapter*{Pré-requisitos 📇}\label{sec-prework}
\addcontentsline{toc}{chapter}{Pré-requisitos 📇}

\markboth{Pré-requisitos 📇}{Pré-requisitos 📇}

Falar sobre os pre-requisitos e intercessões entre as outras disciplinas
e áreas do conhecimento\ldots{}

\begin{tcolorbox}[enhanced jigsaw, breakable, left=2mm, arc=.35mm, leftrule=.75mm, opacitybacktitle=0.6, colframe=quarto-callout-important-color-frame, toptitle=1mm, rightrule=.15mm, colback=white, coltitle=black, title=\textcolor{quarto-callout-important-color}{\faExclamation}\hspace{0.5em}{Tip \ref*{tip-prompt}: Chats de IA para seus trabalhos acadêmicos}, opacityback=0, titlerule=0mm, bottomtitle=1mm, colbacktitle=quarto-callout-important-color!10!white, toprule=.15mm, bottomrule=.15mm]

\quartocallouttip{tip-prompt} 

Os resumos das bibliografias, mapas mentais, ou qualquer estrutura
analítica do conteúdo da disciplina, que apresento em cada uma das
seções, foram elaborados com o auxílio do Gemini 2.5 Pro, seja pelo o
\href{https://gemini.google.com/}{webapp do Google}, pelo
\href{https://notebooklm.google.com/}{NotebookLM} ou
\href{https://aistudio.google.com/}{Google AI Studio} da
Microsoft.\vspace{0.5em}

\end{tcolorbox}

\begin{tcolorbox}[enhanced jigsaw, breakable, left=2mm, arc=.35mm, leftrule=.75mm, opacitybacktitle=0.6, colframe=quarto-callout-caution-color-frame, toptitle=1mm, rightrule=.15mm, colback=white, coltitle=black, title=\textcolor{quarto-callout-caution-color}{\faFire}\hspace{0.5em}{Não confie cegamente na IA}, opacityback=0, titlerule=0mm, bottomtitle=1mm, colbacktitle=quarto-callout-caution-color!10!white, toprule=.15mm, bottomrule=.15mm]

Eu simplesmente copio e colo os resultados do meu chatboot favorito para
compilar notas de leituras, resumir bibliografia, ou de maneira geral,
escrever textos sobre determinado assunto? \textbf{NÃO}. Após a saída do
chatboot eu reviso o texto e faço ajustes, que somente são possíveis,
porque li antes ou tenho bastante conhecimento do material que estou
interagindo. A despeito do App de IA fazer um bom serviço nesse sentido,
ele ainda comete muitos deslizes. Deslizes esses que você não pode
deixar passar num texto público para fins de ensino, e somente seriam
captados a partir da leitura do material ou sendo conhecedor do assunto
abordado.

Comentar sobre o código de conduto para uso de IA do Ricardo\ldots{}

\end{tcolorbox}

\bookmarksetup{startatroot}

\chapter*{Cronograma 📅}\label{sec-schedule}

\markboth{Cronograma 📅}{Cronograma 📅}

\textbf{Em construção\ldots{}}

Planejamento dos dias (📅) e horários das aulas (⏲️), conforme a ementa
do curso. De uma forma geral, as aulas presenciais da disciplina
ocorrerão às sexta-feiras das 19:00 às 22:30 durante o período
02/06/2025 a 29/09/2025.

Na seção de cada uma das aulas temos materiais adicionais para o
respectivo conteúdo. Quando disponível, por aqui, poderás acessar os
slides utilizados nas aulas (🗣️), aulas gravadas ou indicações de vídeo
(🎥{]} e leituras básica sobre os conteúdos (📓).

\begin{longtable}[]{@{}lcc@{}}
\toprule\noalign{}
Aula/Conteúdo & Data & Material Principal \\
\midrule\noalign{}
\endhead
\bottomrule\noalign{}
\endlastfoot
Capítulo~\ref{sec-intro} & 📅04/06/24⏲️19:00 & \href{}{🗣🎥📓} \\
Capítulo~\ref{sec-giro} & 📅06/06/24⏲️19:00 & \href{}{🗣🎥📓} \\
Capítulo~\ref{sec-caixa} & 📅11/06/24⏲️19:00 & \href{}{🗣🎥📓} \\
Capítulo~\ref{sec-credito} & 📅13/06/24⏲️19:00 & \href{}{🗣🎥📓} \\
Capítulo~\ref{sec-estoque} & 📅18/06/24⏲️19:00 & \href{}{🗣🎥📓} \\
Capítulo~\ref{sec-aval} & 📅20/06/24⏲️19:00 & \href{}{🗣🎥📓} \\
& 📅25/06/24⏲️19:00 & \href{}{🗣🎥📓} \\
& 📅27/06/24⏲️19:00 & \href{}{🗣🎥📓} \\
\end{longtable}

\bookmarksetup{startatroot}

\chapter{Introdução à Finanças Corporativas}\label{sec-intro}

\bookmarksetup{startatroot}

\chapter{Capital de Giro}\label{sec-giro}

\bookmarksetup{startatroot}

\chapter{Administração do Caixa}\label{sec-caixa}

\bookmarksetup{startatroot}

\chapter{Administração de Recebíveis}\label{sec-credito}

\bookmarksetup{startatroot}

\chapter{Administração de Estoques}\label{sec-estoque}

\bookmarksetup{startatroot}

\chapter{Projetos em Grupo}\label{sec-aval}

\bookmarksetup{startatroot}

\chapter*{Referências}\label{referuxeancias}
\addcontentsline{toc}{chapter}{Referências}

\markboth{Referências}{Referências}

\phantomsection\label{refs}
\begin{CSLReferences}{1}{0}
\bibitem[\citeproctext]{ref-assafneto2014}
Assaf Neto, A. (2014). \emph{Finan{ç}as Corporativas e Valor} (7. ed.).
Atlas.

\bibitem[\citeproctext]{ref-brealey2013}
Brealey, R. A., Myers, S. C., \& Allen, F. (2013). \emph{Princ{í}pios de
Finan{ç}as Corporativas} (10. ed.). AMGH.

\bibitem[\citeproctext]{ref-dogucu2022}
Dogucu, M., \& Çetinkaya-Rundel, M. (2022). Tools and Recommendations
for Reproducible Teaching. \emph{Journal of Statistics and Data Science
Education}, \emph{30}(3), 251--260.
\url{https://doi.org/10.1080/26939169.2022.2138645}

\bibitem[\citeproctext]{ref-gitman2010}
Gitman, L. J. (2010). \emph{Princ{í}pios de Administra{ç}{ã}o
Financeira} (12. ed.). Pearson Prentice Hall.

\bibitem[\citeproctext]{ref-matias2007}
Matias, A. M. (Ed.). (2007). \emph{Finan{ç}as Corporativas de Curto
Prazo, Volume 1: A Gest{ã}o Do Valor Do Capital de Giro}. Atlas.

\bibitem[\citeproctext]{ref-ross2015}
Ross, S. A., Westerfield, R. W., Jaffe, J., \& Lamb, R. (2015).
\emph{Administra{ç}{ã}o Financeira} (10. ed.). AMGH.

\end{CSLReferences}


\backmatter

\end{document}
